\documentclass{sig-alternate-2013}
\usepackage{lmodern}
\usepackage[T1]{fontenc}
\usepackage{amssymb}
\usepackage{booktabs}
\usepackage{dsfont}
\usepackage{hyperref}
\usepackage{morefloats}
\usepackage[numbers,square,sort&compress]{natbib}
\renewcommand{\refname}{References}
\renewcommand{\bibsection}{\section{References}}
\usepackage[ruled]{algorithm2e} % must be loaded after natbib
\usepackage{subcaption}

\newcommand\mycommfont[1]{\footnotesize{#1}}
\SetCommentSty{mycommfont}
\usepackage{mdwlist}

%\usepackage[show]{notes-alt}
\newcommand{\TODO}{{\bf \sf TODO: }}
\newcommand{\RFC}{{\bf \sf FEEDBACK PLEASE: }}

\DeclareMathOperator*{\argmin}{arg\,min}

\newcommand{\Sam}{{\cal S}}
\newcommand{\Ds}{{\cal D}}
\newcommand\Itm{{\cal I}}
\newcommand\TOPK{\mathsf{TOPK}}
\newcommand\FI{\mathsf{FI}}
\newcommand\CI{\mathsf{CI}}
\newcommand\AR{\mathsf{AR}}
\newcommand\VC{\mathsf{VC}}
\newcommand\EVC{\mathsf{EVC}}
\newcommand\range{\mathcal{R}}
\newcommand\expectation{\mathbb{E}}
\newcommand\var{\mathrm{Var}}
\newcommand\CPP{C\nolinebreak[4]\hspace{-.05em}\raisebox{.4ex}{\relsize{-3}{\textbf{++}}}11}
\newcommand\tid{\mathsf{tid}}
\newcommand\indicator{\mathds{1}}
\newcommand\lgi{\mathsf{lgi}}

\newcommand{\spara}[1]{\smallskip\noindent{\bf #1.}}
\newcommand{\para}[1]{\noindent{\bf #1.}}

\newtheorem{corollary}{Corollary}
\newtheorem{lemma}{Lemma}
\newtheorem{theorem}{Theorem}
\newtheorem{fact}{Fact}
\newtheorem{claim}{Claim}

%\theoremstyle{definition}
\newtheorem{definition}{Definition}

\begin{document}

\newfont{\mycrnotice}{ptmr8t at 7pt}
\newfont{\myconfname}{ptmri8t at 7pt}
\let\crnotice\mycrnotice%
\let\confname\myconfname%

\CopyrightYear{2015}
\permission{Permission to make digital or hard copies of all or part of this work for personal or classroom use is granted without fee provided that copies are not made or distributed for profit or commercial advantage and that copies bear this notice and the full citation on the first page. Copyrights for components of this work owned by others than ACM must be honored. Abstracting with credit is permitted. To copy otherwise, or republish, to post on servers or to redistribute to lists, requires prior specific permission and/or a fee. Request permissions from Permissions@acm.org.}
\conferenceinfo{KDD'15,}{August 10-13, 2015, Sydney, NSW, Australia.}
\copyrightetc{\copyright~2015 ACM. ISBN \the\acmcopyr}
\crdata{978-1-4503-3664-2/15/08\ ...\$15.00.\\
DOI: http://dx.doi.org/10.1145/2783258.2789984
}

\clubpenalty=10000
\widowpenalty = 10000

\numberofauthors{2}

\title{VC-Dimension and Rademacher Averages:\\From Statistical Learning Theory to Sampling Algorithms}

\author{
\alignauthor
Matteo Riondato\\
       \affaddr{Dept.~of Computer Science}\\
       \affaddr{Brown University}\\
       \affaddr{Providence, RI 02912}\\
       \email{matteo@cs.brown.edu}
\alignauthor
Eli Upfal\\
       \affaddr{Dept.~of Computer Science}\\
       \affaddr{Brown University}\\
       \affaddr{Providence, RI 02912}\\
       \email{eli@cs.brown.edu}
}
\date{\today}

\maketitle

\begin{abstract}
	\emph{Rademacher Averages} and the \emph{Vapnik-Chervonenkis dimension} are
	fundamental concepts from statistical learning theory. They allow to study
	simultaneous deviation bounds of empirical averages from their expectations
	for classes of functions, by considering properties of the functions, of
	their domain (the dataset), and of the sampling process. In this tutorial,
	we survey the use of Rademacher Averages and the VC-dimension in
	sampling-based algorithms for graph analysis and pattern mining. We start
	from their theoretical foundations at the core of machine learning, then
	show a generic recipe for formulating data mining problems in a way that
	allows to use these concepts in efficient randomized algorithms for those
	problems. Finally, we show examples of the application of the recipe to
	graph problems (connectivity, shortest paths, betweenness centrality) and
	pattern mining.  Our goal is to expose the usefulness of these techniques
	for the data mining researcher, and to encourage research in the area.
\end{abstract}

\category{H.2.8}{Database Management}{Database Applications}[Data
mining]
\category{G.3}{Probability and Statistics}[Probabilistic algorithms (including
Monte Carlo)]

\keywords{Betweenness Centrality; Frequent Itemsets; Graph
Mining; Pattern Mining; Randomized Algorithms; Tutorial}

\section{Introduction}\label{introduction}
\emph{Random sampling} is a natural technique to speed up the execution of
algorithms on very large datasets~\citep{CormodeD14}. The results obtained by
analyzing only a random sample of the dataset are an approximation of the exact
solution.  When only a single value must be computed, the trade-off between the
size of the sample and the accuracy of the approximation can be studied through
probabilistic bounds for the deviation of the quantity of interest in
the sample from its exact value in the dataset, e.g., the Chernoff-Hoeffding
bounds~\citep{Hoeffding63}. In many data mining problems, the number of
quantities of interest can be extremely large (e.g., betweenness centrality
requires to compute one quantity for each node in a
graph~\citep{Brandes01,RiondatoK14}, and Frequent Pattern Mining requires the
computation of a potentially exponential number of quantities). In these cases,
\emph{uniform} (i.e., \emph{simultaneous}) bounds to the deviations of all
quantities are needed in order to have good approximations of all the values of
interest. Classical techniques like the Union bound~\citep{MitzenmacherU05} are
insufficient because excessively loose due to their worst-case assumptions that
do not hold in many data mining problems, e.g., the assumption that the
quantities of interest are independent from each other. \emph{Rademacher
Averages}~\citep{BoucheronBL05} and the \emph{Vapnik-Chervonenkis
dimension}~\citep{Vapnik99} have been developed by the statistical learning
theory community to study the (rate of) convergence of the empirical risk of a
learned function to its expectation. One of the goals of this tutorial is to
show that these techniques are very flexible and powerful and their field of use
is much broader. In particular, they overcome the weakness of the Union bound:
they obtain much stricter uniform deviation bounds by taking into account the
nature of the problem (i.e., of the quantities of interest) and properties of
the dataset and of the sampling process. They have been used with success in
the analysis of sampling algorithms for data and graph analysis problems on very
large
datasets~\citep{RiondatoU14,RiondatoU15,RiondatoK14,RiondatoV14,AbrahamDFGW11,KleinbergSS08,BronnimannG95}.

\section{Outline}\label{sec:outline}
The tutorial is structured as follows.

{\bf Introduction.} We start with a short introduction about the use of random
sampling in data mining, discussing its advantages and the challenges for the
algorithm designer.  The goal is to lay forward the key questions that will be
answered in the rest of the tutorial. In particular, we start with an example
involving Frequent Itemset Mining through
sampling~\citep{RiondatoU14,RiondatoU15}, showing the limitations of the Union
bound in solving this problem. By generalizing our settings to general data
mining problem, we then introduce the key problem of learning, known as the
\emph{Glivenko-Cantelli problem for classes of functions}~\citep{Vapnik99},
which clarifies the strong connection with the area of statistical learning
theory.

\paragraph*{First Part: Theoretical Foundations}
The theoretical foundations are presented in the first part, where we introduce
the Rademacher Averages~\citep{BoucheronBL05,Koltchinskii01}  and the
VC-dimension~\citep{VapnikC71}, showing how they allow to answer the questions
posed in the introduction, and how they are related to each other. In
particular, we show the key theorems that allow to compute an upper bound to the
sample size sufficient to obtain a high-quality approximation of the quantities
of interest, uniformly. The bounds depend linearly on the Rademacher averages
and on the VC-dimension. We then focus on computing, estimating, and bounding
the Rademacher averages and the VC-dimension, which is a key
step in the process of using them to develop algorithms. We show a number of
basic examples of classes of functions with finite and infinite VC-dimension and
discuss different techniques for developing analytical bounds and empirical
estimations. The examples will range from toy examples (e.g., axis-aligned
rectangles, half-spaces, and sinusoidal functions) to much more complex
instances that are presented in research papers (e.g., graph neighborhood
functions, neural networks~\citep{AnthonyB99}, and shortest
paths~\citep{AbrahamDFGW11}).

\paragraph*{Second Part: Applications}
The second part focuses on showing how to use Rademacher averages and
VC-dimension to develop sampling-based algorithms for data and graph mining
problems. We start by presenting a generic recipe for developing such
algorithms, which eases the application of the techniques and the analysis of
the algorithms. We then show a number of examples of application of this
technique for different graph and data analysis problems, including network
connectivity~\citep{KleinbergSS08}, shortest paths
algorithms~\citep{AbrahamDFGW11}, betweenness centrality
computation~\citep{RiondatoK14}, and frequent pattern
mining~\citep{RiondatoU14,RiondatoV14,RiondatoU15}, and set
covering~\citep{BronnimannG95}.

\paragraph*{Third part: Advanced Material} In the third part, we will focus on more
advanced material, to encourage the audience to further explore the field of
statistical learning theory, and to stimulate discussion and research on using
the results from this field to develop data mining algorithms. Specifically, we
will discuss: PAC-Bayesian bounds~\citep{BoucheronBL05,ShalevSBD14}, which show
a connection between the typical frequentist approach followed in statistical
learning theory to the Bayesian probabilistic approach and may be useful for
data mining algorithms on uncertain or probabilistic data, and a selection of the
extensions of VC-dimension to real-valued or non-binary functions, including
pseudodimension~\citep{Pollard84}, Natarajan dimension, and fat-shattering
dimension~\citep{KearnsS90}.

\section{Website}\label{sec:website}
We set up a mini-website (\url{http://bigdata.cs.brown.edu/vctutorial}) with
links to the slides that we use for the presentation, and a bibliography of
theoretical and application-oriented works about VC-dimension and Rademacher
Averages.

\section{Acknowledgments}\label{sec:ack}
This work was supported by NSF grant IIS-1247581 and NIH grant R01-CA180776.

%\bibliographystyle{abbrvnat}
%\bibliography{tutorialbib}
%\iffalse
\begin{thebibliography}{17}
\providecommand{\natexlab}[1]{#1}
\providecommand{\url}[1]{\texttt{#1}}
\expandafter\ifx\csname urlstyle\endcsname\relax
  \providecommand{\doi}[1]{doi: #1}\else
  \providecommand{\doi}{doi: \begingroup \urlstyle{rm}\Url}\fi

\bibitem[Abraham et~al.(2011)Abraham, Delling, Fiat, Goldberg, and
  Werneck]{AbrahamDFGW11}
I.~Abraham, D.~Delling, A.~Fiat, A.~V. Goldberg, and R.~F. Werneck.
\newblock {VC}-dimension and shortest path algorithms.
\newblock ICALP'11, 2011.

\bibitem[Anthony and Bartlett(1999)]{AnthonyB99}
M.~Anthony and P.~L. Bartlett.
\newblock \emph{Neural Network Learning - Theoretical Foundations}.
\newblock Cambridge University Press, 1999.

\bibitem[Boucheron et~al.(2005)Boucheron, Bousquet, and Lugosi]{BoucheronBL05}
S.~Boucheron, O.~Bousquet, and G.~Lugosi.
\newblock Theory of classification : A survey of some recent advances.
\newblock \emph{{ESAIM}: Probability and Statistics}, 9:\penalty0 323--375,
  2005.

\bibitem[Brandes(2001)]{Brandes01}
U.~Brandes.
\newblock A faster algorithm for betweenness centrality.
\newblock \emph{J. Math. Sociol.}, 25\penalty0 (2):\penalty0 163--177, 2001.

\bibitem[Br{\"o}nnimann and Goodrich(1995)]{BronnimannG95}
H.~Br{\"o}nnimann and M.~Goodrich.
\newblock Almost optimal set covers in finite vc-dimension.
\newblock \emph{Discrete \& Computational Geometry}, 14\penalty0 (1):\penalty0
  463--479, 1995.

\bibitem[Cormode and Duffield(2014)]{CormodeD14}
G.~Cormode and N.~Duffield.
\newblock Sampling for big data: A tutorial.
\newblock ACM KDD'14, 2014.

\bibitem[Hoeffding(1963)]{Hoeffding63}
W.~Hoeffding.
\newblock Probability inequalities for sums of bounded random variables.
\newblock \emph{J.~American Statistical Assoc.}, 58\penalty0 (301):\penalty0
  13--30, 1963.

\bibitem[Kearns and Schapire(1990)]{KearnsS90}
M.~J. Kearns and R.~E. Schapire.
\newblock Efficient distribution-free learning of probabilistic concepts.
\newblock IEEE FOCS'90, 1990.

\bibitem[Kleinberg et~al.(2008)Kleinberg, Sandler, and Slivkins]{KleinbergSS08}
J.~M. Kleinberg, M.~Sandler, and A.~Slivkins.
\newblock Network failure detection and graph connectivity.
\newblock \emph{SIAM J. Comput.}, 38\penalty0 (4):\penalty0 1330--1346, 2008.

\bibitem[Koltchinskii(2001)]{Koltchinskii01}
V.~Koltchinskii.
\newblock Rademacher penalties and structural risk minimization.
\newblock \emph{IEEE Trans.~Inf.~Theory}, 47\penalty0 (5):\penalty0 1902--1914,
  July 2001.

\bibitem[Mitzenmacher and Upfal(2005)]{MitzenmacherU05}
M.~Mitzenmacher and E.~Upfal.
\newblock \emph{Probability and Computing: Randomized Algorithms and
  Probabilistic Analysis}.
\newblock Cambridge University Press, 2005.

\bibitem[Pollard(1984)]{Pollard84}
D.~Pollard.
\newblock \emph{Convergence of stochastic processes}.
\newblock Springer-Verlag, 1984.

\bibitem[Riondato and Kornaropoulos(2014)]{RiondatoK14}
M.~Riondato and E.~M. Kornaropoulos.
\newblock Fast approximation of betweenness centrality through sampling.
\newblock ACM WSDM'14, 2014.

\bibitem[Riondato and Upfal(2014)]{RiondatoU14}
M.~Riondato and E.~Upfal.
\newblock Efficient discovery of association rules and frequent itemsets
  through sampling with tight performance guarantees.
\newblock \emph{ACM Trans. Knowl. Disc. from Data}, 8\penalty0 (4):\penalty0
  20, 2014.

\bibitem[Riondato and Upfal(2015)]{RiondatoU15}
M.~Riondato and E.~Upfal.
\newblock Mining frequent itemsets through progressive sampling with rademacher
  averages.
\newblock ACM KDD'15.

\bibitem[Riondato and Vandin(2014)]{RiondatoV14}
M.~Riondato and F.~Vandin.
\newblock Finding the true frequent itemsets.
\newblock SIAM SDM'14.

\bibitem[Shalev-Shwartz and Ben-David(2014)]{ShalevSBD14}
S.~Shalev-Shwartz and S.~Ben-David.
\newblock \emph{Understanding Machine Learning: From Theory to Algorithms}.
\newblock Cambridge University Press, 2014.

\bibitem[Vapnik(1999)]{Vapnik99}
V.~N. Vapnik.
\newblock \emph{The Nature of Statistical Learning Theory}.
\newblock Springer-Verlag, 1999.

\bibitem[Vapnik and Chervonenkis(1971)]{VapnikC71}
V.~N. Vapnik and A.~J. Chervonenkis.
\newblock On the uniform convergence of relative frequencies of events to their
  probabilities.
\newblock \emph{Theory of Probability and its Applications}, 16\penalty0
  (2):\penalty0 264--280, 1971.
\end{thebibliography}
%\fi
\end{document}
