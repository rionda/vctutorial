\documentclass[11pt]{article}
\usepackage{fullpage}
\usepackage{lmodern}
\usepackage{textcomp}
\usepackage[T1]{fontenc}
\usepackage{amsmath}
\usepackage{amssymb}
\usepackage{amsfonts}
\usepackage[numbers,square,sort&compress]{natbib}
\usepackage{hyperref}
\usepackage{mdwlist}

\setlength{\bibsep}{0pt plus 0.3ex}

\renewcommand{\labelenumi}{\arabic{enumi}.}
\renewcommand{\labelenumii}{\arabic{enumi}.\arabic{enumii}}

\title{VC-Dimension and Rademacher Averages:\\From Statistical Learning Theory to
	Sampling Algorithms \\ {\large -- Proposal for a KDD tutorial -- }}
\author{Matteo Riondato \qquad Eli Upfal \\ Department of Computer
	Science -- Brown University \\
	\textsf{\{matteo,eli\}@cs.brown.edu}}
\date{}
\begin{document}
\maketitle

\begin{abstract}
    %Abstract (up to 150 words)
	\emph{Rademacher Averages} and the \emph{Vapnik-Chervonenkis dimension} are fundamental
concepts from statistical learning theory. They allow to study simultaneous deviation bounds
of empirical averages from their expectations for classes of functions, by
considering properties of the problem, of the dataset, and of the
sampling process. In this tutorial, we survey the use of Rademacher Averages and
the VC-dimension for developing sampling-based algorithms for graph analysis and
pattern mining. We start from their theoretical foundations at the core of
machine learning, then show a generic recipe for formulating data mining
problems in a way that allows using these concepts in the analysis of efficient
randomized algorithms for those problems. Finally, we show examples of the
application of the recipe to graph problems (connectivity, shortest paths,
betweenness centrality) and pattern mining. Our goal is to expose the usefulness
of these techniques for the data mining researcher, and to encourage research in
the area.
\end{abstract}

\section*{Introduction}
\emph{Random sampling} is a natural technique to speed up the execution of
algorithms on very large datasets. The results obtained by analyzing only a
random sample of the dataset are an approximation of the exact solution.  When
only a single value must be computed, the trade-off between the size of the
sample and the accuracy of the approximation can be studied through
probabilistic bounds (e.g., the Chernoff-Hoeffding bounds) for the deviation of
the quantity of interest in the sample from its exact value in the dataset. In
many classical data mining problems, the number of quantities of interest can be
extremely large (e.g., betweenness centrality requires to compute one quantity
for each node in a graph). In these cases, \emph{uniform} (i.e.,
\emph{simultaneous}) bounds to the deviations of all quantities are needed.
Classical techniques like the Union bound, are insufficient because excessively
loose due to their worst-case assumptions that do not hold in many data mining
problems. \emph{Rademacher Averages} and the \emph{Vapnik-Chervonenkis dimension} have
been developed to overcome this issue: they obtain much stricter uniform
deviation bounds by taking into account the nature of the problem and properties
of the dataset and of the sampling process. They have been used with success in
the analysis of sampling algorithms for data and graph analysis problems on very
large datasets.

\section*{Target Audience and Prerequisites}
%Proposals must clearly identify the intended audience for the tutorial (e.g.,
%novice users of statistical techniques, or expert researchers in text mining).
%What background will be required of the audience? Why is this topic
%important/interesting to the KDD community? What is the benefit to
%participants?
The target audience of our tutorial are data mining and machine learning
researchers with an interest in modern statistics and in effective use of data
through the application of powerful theoretical results. Researchers whose focus
is in pattern mining and graph mining should be particularly interested, as should
the part of the audience with an interest in the theory of machine learning. In
designing our tutorial we effectively strived in strengthen the connection
between these fields: we start from theoretical concepts and results that arise
in the classical learning context of classification and show how they can be
used to develop practical efficient algorithms for important data mining
problems. For the attending machine learning researchers, it would be a chance
to be exposed to recent theoretical material in their field that has only
recently started to be organized in books~\citep{MohriRT12,ShalevSBD14}, and see
how it can be applied outside machine learning. At the same time, the part of
the audience more interested in data mining can learn about techniques from the
neighbor field of machine learning that can be used to develop algorithms for
their favorite knowledge discovery problems. For these reasons, we believe that
our tutorial can be of great interest to most KDD attendees.

We do not require or assume that any specific existing knowledge from the
audience. The tutorial is designed for an audience of computer scientists who
have a general idea of the problems and challenges in data mining and a basic
understanding of probability. We believe that any advanced undergraduate student
in computer science would be able to productively follow our tutorial, and we
will actively engage with the audience to adapt our pace and our presentation
style to ensure that every attendee can benefit from our tutorial.

\section*{Tutorial Content and Outline}
%Outline of the tutorial. Enough material should be included to provide a sense
%of both the scope of material to be covered and the depth to which it will be
%covered. The more details that can be provided, the better (up to and including
%links to the actual slides). Note that the tutors should NOT focus exclusively
%on their own research results. A KDD tutorial is not meant to be a forum for
%promoting one's research or product.


We plan for a three-hours tutorial. We start with a short introduction
about the use of random sampling in data mining, discussing its advantages and
the challenges for the algorithm designer. The goal is to lay forward the key
questions that will be answered in the rest of the tutorial. In particular, we
introduce the key problem of learning, known as the Glivenko-Cantelli problem
for classes of functions, and then show the limitations of the Union bound in
solving this problem and ask how to overcome them in the settings arising from
using sampling for data mining problems.

\medskip
The theoretical foundations are presented in the first part, where we introduce
the Rademacher Average and the VC-dimension, showing how they allow to answer
the questions posed in the introduction, and how they are related to each other.
We then focus on computing, estimating, and bounding these quantities, which is
a key step in the process of using them to develop algorithms. We show a number
of basic examples of classes of functions with finite and infinite VC-dimension
and discuss different techniques for developing analytical bounds and empirical
estimations. The examples will range from toy examples to understand the
concepts (e.g., axis-aligned rectangles, half-spaces, and sinusoidal functions)
to much more complex examples that are presented in research papers (e.g.,
graph neighborhood functions, neural networks, and shortest paths).

\medskip
The second part focuses on showing how to use Rademacher averages and
VC-dimension to develop sampling-based algorithms for data and graph mining
problems. We start by presenting a generic recipe for developing such
algorithms, which makes it easier to applying the techniques and perform the
analysis. We then show a number of examples of application of this technique for
different graph and data analysis problems, including network connectivity,
shortest paths algorithms, betweenness centrality computation, and frequent
pattern mining, and set covering. In this part of the tutorial, we will
partially discuss research done by one or both the tutorial proposers. This is
quite unavoidable, as we were among the firsts to explore the use of these
techniques in data mining.  Naturally, a number of works by other researchers
will also be discussed here and in the other parts of the tutorial.

\medskip
In the third part, we will focus on more advanced material, to encourage the
audience to further explore the field of statistical learning theory, and to
stimulate discussion and research on using the results from this field to
develop data mining algorithms. Specifically, we will discuss: {\em 1.} PAC-Bayesian bounds, which shows a connection between the typical frequentist
approach followed in statistical learning theory to the Bayesian probabilistic
approach, {\em 2.} the connection between VC-dimension and the Minimum
Description Length principle from information theory, and {\em 3.} a selection
of the extensions of VC-dimension to real-valued or non-binary functions,
including pseudodimension, Natarajan dimension, and fat-shattering dimension.

\medskip
The following is a preliminary outline of the tutorial.

\begin{enumerate*}
	\item Introduction and Theoretical Foundations (1 hour)
		\begin{enumerate*}
			\item Sampling and Data Mining: a happy marriage?
			\item The Glivenko-Cantelli problem: uniform convergence of
				classes of functions~\citep{Vapnik99}
			\item Beyond the Union Bound: Rademacher
				averages~\citep{Koltchinskii01} and the Vapnik-Chervonenkis
				dimension~\citep{VapnikC71}. The sampling theorems~\citep{AlonS08}
			\item Computing the VC-dimension is hard~\citep{PapadimitriouY96}.
				Estimating the VC-dimension~\citep{VapnikLLC94,ShaoCL00}.
				Bounding the VC-dimension: axis-aligned rectangles, sine functions,
				paths and stars in a graph~\citep{KranakisKRUW97}
		\end{enumerate*}
	\item Applications to Graph and Pattern Mining (1 hour and 30 minutes)
		\begin{enumerate*}
			\item A generic recipe for sampling-based data mining algorithms
			\item Applications to Graph Mining: identifying network
				disruption~\citep{KleinbergSS08}, shortest path
				algorithms~\citep{AbrahamDFGW11}, betweenness centrality~\citep{RiondatoK14}
			\item Applications to Pattern Mining: static and progressive
				sampling algorithms for frequent itemsets~\citep{RiondatoU14,RiondatoU15-ext}
			\item Other applications: set covering~\citep{BronnimannG94}
		\end{enumerate*}
	\item Recent developments and advanced topics (30 minutes)
		\begin{enumerate*}
			\item PAC-Bayesian bounds~\citep{BoucheronBL05,ShalevSBD14}
			\item Statistical Risk Minimization and the connection with the Minimum
				Description Length principle~\citep{Vapnik99}
			\item Beyond binary functions: pseudodimension~\citep{Pollard1984}, fat-shattering
				dimension~\citep{KearnsS90}, and related measures~\citep{MohriRT12}
		\end{enumerate*}
\end{enumerate*}

\section*{Time Plan for Tutorial Materials Preparation}
We plan to create a mini-website for the tutorial at
\url{http://bigdata.cs.brown.edu/vctutorial/}. The website will contain the
abstract of the tutorial, a more detailed outline with short a description of
each item of the outline, a full list of references complete with links to
electronic editions, and naturally the slides used in the tutorial. A
preliminary version of the website will be available 15 days after the tutorial
is accepted. We plan to work on it and enrich the contents continuously. A
preliminary version of the slides will be available 30 days before the
conference, or in any case by any deadline given to us by the conference
organizers, and the final version will be available 15 days before the
conference.

\section*{Tutors}
This tutorial is developed by Matteo Riondato and Eli Upfal.

\medskip
\noindent\textsc{Matteo Riondato} is a postdoctoral research associate at Brown
University, USA, supervised by Prof.~Eli Upfal. He received his Ph.D.~from Brown
in May 2014, with a dissertation on sampling-based randomized algorithms for
data analytics, which received the Best Student Poster Award at SIAM SDM 2014.
He presented a nectar talk about modern sampling algorithms at ECML PKDD 2014.
His research focuses on exploiting theoretical results for practical algorithms
in pattern and graph mining.

\medskip
\noindent\textsc{Eli Upfal} is a professor of computer science at Brown
University, where he was also the department chair from 2002 to 2007. Prior to
joining Brown in 1998, he was a researcher and project manager at the IBM
Almaden Research Center in California, and a professor of Applied Mathematics
and Computer Science at the Weizmann Institute of Science in Israel. Upfal's
research focuses on the design and analysis of algorithms. In particular he is
interested in randomized algorithms, probabilistic analysis of algorithms, and
computational statistics, with applications ranging from combinatorial and
stochastic optimization to routing and communication networks, computational
biology, and computational finance. Upfal is a fellow of the IEEE and the
Association for Computing Machinery (ACM). He received the IBM Outstanding
Innovation Award, and the IBM Research Division Award. His work at Brown has
been funded in part by the National Science Foundation (NSF), the Defense
Advanced Research Projects Agency (DARPA), the Office of Naval Research (ONR),
and the National Institute of Health (NIH). He is co-author of a popular
textbook ``Probability and Computing: Randomized Algorithms and Probabilistic
Analysis'' (with M. Mitzenmacher, Cambridge University Press 2005).

\subsection*{Contact Information}
Matteo Riondato and Eli Upfal

\textsf{\{matteo,eli\}@cs.brown.edu}

Box 1910, 115 Waterman Street, Providence, RI 02912, USA

Tel.: +14018631000

\section*{Previous Venues and Similar Tutorials}
The tutorial was never presented before. Matteo Riondato gave a nectar
talk with limited overlap with the tutorial content at ECML PKDD
2014~\citep{Riondato14}. The tutorial we propose is much richer in content and
more structured in its organization.

A tutorial on sampling, but \emph{not covering any of the material in our proposed
content}, was presented at ACM KDD 2014~\citep{CormodeD14}. It can be seen as
complementary to ours, introducing different techniques for different
applications (mostly in streaming settings). We believe that the fact that a
tutorial on sampling was presented last year at KDD reinforces the fact that
machine learning and data mining researchers should be interested in
sampling-related techniques. Moreover VC-dimension can be used for much more
than sampling (for example, for computing set coverings). We believe that data
mining research can greatly benefit from the rich theory, techniques, and tools
developed in the context of statistical learning. This tutorial aims to spread
the knowledge of these concepts to the wider data mining community. This goal
sets this tutorial apart from the KDD 2014 sampling tutorial.

\begin{thebibliography}{20}
\providecommand{\natexlab}[1]{#1}
\providecommand{\url}[1]{\texttt{#1}}
\expandafter\ifx\csname urlstyle\endcsname\relax
  \providecommand{\doi}[1]{doi: #1}\else
  \providecommand{\doi}{doi: \begingroup \urlstyle{rm}\Url}\fi

\bibitem[Abraham et~al.(2011)Abraham, Delling, Fiat, Goldberg, and
  Werneck]{AbrahamDFGW11}
I.~Abraham, D.~Delling, A.~Fiat, A.~V. Goldberg, and R.~F. Werneck.
\newblock {VC}-dimension and shortest path algorithms.
\newblock ICALP'11, 2011. 

\bibitem[Alon and Spencer(2008)]{AlonS08}
N.~Alon and J.~H. Spencer.
\newblock \emph{The Probabilistic Method}.
\newblock Interscience Series in Discrete Mathematics and Optimization. John
  Wiley {\&} Sons, Hoboken, NJ, USA, third edition, 2008.

\bibitem[Boucheron et~al.(2005)Boucheron, Bousquet, and Lugosi]{BoucheronBL05}
S.~Boucheron, O.~Bousquet, and G.~Lugosi.
\newblock Theory of classification : A survey of some recent advances.
\newblock \emph{{ESAIM}: Probability and Statistics}, 9:\penalty0 323--375,
  2005.

\bibitem[Br\"{o}nnimann and Goodrich(1994)]{BronnimannG94}
H.~Br\"{o}nnimann and M.T. Goodrich.
\newblock Almost Optimal Set Covers in Finite VC-Dimension
\newblock SCG'94, 1994.

\bibitem[Cormode and Duffield(2014)]{CormodeD14}
G.~Cormode and N.~Duffield.
\newblock Sampling for Big Data: A tutorial. 
\newblock KDD'14, 2014.

\bibitem[Kearns and Schapire(1990)]{KearnsS90}
M.~J. Kearns and R.~E. Schapire.
\newblock Efficient distribution-free learning of probabilistic concepts.
\newblock FOCS'90, 1990.

\bibitem[Kleinberg et~al.(2008)Kleinberg, Sandler, and Slivkins]{KleinbergSS08}
J.~M. Kleinberg, M.~Sandler, and A.~Slivkins.
\newblock Network failure detection and graph connectivity.
\newblock \emph{SIAM J. Comput.}, 38\penalty0 (4):\penalty0 1330--1346, 2008.

\bibitem[Koltchinskii(2001)]{Koltchinskii01}
V.~Koltchinskii.
\newblock Rademacher penalties and structural risk minimization.
\newblock \emph{IEEE Trans.~Inf.~Theory}, 47\penalty0 (5):\penalty0 1902--1914,
  July 2001.

\bibitem[Kranakis et~al.(1997)Kranakis, Krizanc, Ruf, Urrutia, and
  Woeginger]{KranakisKRUW97}
E.~Kranakis, D.~Krizanc, B.~Ruf, J.~Urrutia, and G.~Woeginger.
\newblock The {VC}-dimension of set systems defined by graphs.
\newblock \emph{Discrete Applied Mathematics}, 77\penalty0 (3):\penalty0
  237--257, 1997.

\bibitem[Mohri et~al.(2012)Mohri, Rostamizadeh, and Talwalkar]{MohriRT12}
M.~Mohri, A.~Rostamizadeh, and A.~Talwalkar.
\newblock \emph{Foundations of machine learning}.
\newblock The MIT Press, 2012.

\bibitem[Papadimitriou and Yannakakis(1996)]{PapadimitriouY96}
C.~H. Papadimitriou and M.~Yannakakis.
\newblock On limited nondeterminism and the complexity of the {V-C} dimension.
\newblock \emph{J. Comput. Syst. Sci.}, 53\penalty0 (2):\penalty0 161--170,
  1996.

\bibitem[Pollard(1984)]{Pollard1984}
D.~Pollard.
\newblock \emph{Convergence of stochastic processes}.
\newblock Springer, 1984.

\bibitem[Riondato(2014)]{Riondato14}
M.~Riondato.
\newblock Sampling-based data mining algorithms: Modern techniques and case
  studies.
\newblock ECML PKDD'14, 2014.

\bibitem[Riondato and Kornaropoulos(2014)]{RiondatoK14}
M.~Riondato and E.~M. Kornaropoulos.
\newblock Fast approximation of betweenness centrality through sampling.
\newblock WSDM'14, 2014.

\bibitem[Riondato and Upfal()]{RiondatoU15-ext}
M.~Riondato and E.~Upfal.
\newblock Mining frequent itemsets through progressive sampling with rademacher
  averages.
\newblock Extended Version.
\newblock URL \url{http://cs.brown.edu/%7Ematteo/papers/progrsamplfi-ext.pdf}.

\bibitem[Riondato and Upfal(2014)]{RiondatoU14}
M.~Riondato and E.~Upfal.
\newblock Efficient discovery of association rules and frequent itemsets
  through sampling with tight performance guarantees.
\newblock \emph{ACM Trans. Knowl. Disc. from Data}, 8\penalty0 (4):\penalty0
  20, 2014.

\bibitem[Shalev-Shwartz and Ben-David(2014)]{ShalevSBD14}
S.~Shalev-Shwartz and S.~Ben-David.
\newblock \emph{Understanding Machine Learning: From Theory to Algorithms}.
\newblock Cambridge University Press, 2014.

\bibitem[Shao et~al.(2000)Shao, Cherkassky, and Li]{ShaoCL00}
X.~Shao, V.~Cherkassky, and W.~Li.
\newblock Measuring the {VC}-dimension using optimized experimental design.
\newblock \emph{Neural Computation}, 12\penalty0 (8):\penalty0 1969--1986,
  2000.

\bibitem[Vapnik(1999)]{Vapnik99}
V.~N. Vapnik.
\newblock \emph{The Nature of Statistical Learning Theory}.
\newblock Statistics for engineering and information science. Springer-Verlag,
  New York, NY, USA, 1999.

\bibitem[Vapnik and Chervonenkis(1971)]{VapnikC71}
V.~N. Vapnik and A.~J. Chervonenkis.
\newblock On the uniform convergence of relative frequencies of events to their
  probabilities.
\newblock \emph{Theory of Probability and its Applications}, 16\penalty0
  (2):\penalty0 264--280, 1971.

\bibitem[Vapnik et~al.(1994)Vapnik, Levin, and Le~Cun]{VapnikLLC94}
V.~N. Vapnik, E.~Levin, and Y.~Le~Cun.
\newblock Measuring the {VC}-dimension of a learning machine.
\newblock \emph{Neural Computation}, 6:\penalty0 851--876, 1994.

\end{thebibliography}

%\bibliographystyle{abbrvnat}
%\bibliography{tutorialbib}
\end{document}
